
\newcount\draft\draft=1 % set to 0 for submission or publication
\documentclass{article}
\usepackage[utf8]{inputenc}
\usepackage{amsmath,amsthm,amssymb,bbm,amsfonts,syntax}
\usepackage[ligature, inference]{semantic}
\usepackage[T1]{fontenc}
\usepackage{mathpartir}
\usepackage{nccmath}
\usepackage{xxx}

\title{Unnamed Language with Linear Algebra Types: \\ Syntax and Semantics}
\author{Dietrich Geisler \and Irene Yoon \and Adrian Sampson}
\date{}

\newcommand{\mat}{\mathsf{mat}_{n_1{\times}n_2}}
\newcommand{\vv}[1]{\mathsf{vec}_{#1}}
\newcommand{\env}[1]{#1,\sigma}
\newcommand{\defas}{\mathrel{::=}}
\newenvironment{leftalign}%
    {\fleqn[5pt]\csname align*\endcsname}%
    {\csname endalign*\endcsname\endfleqn}
\newcommand{\alt}{\:|\:}

\mathlig{->}{\rightarrow}
\mathlig{|-}{\vdash}
\mathlig{=>}{\Rightarrow}

\begin{document}
\maketitle

\mathligson

\section{Syntax}

Literals in our language can be scalar numbers, vectors, or matrices:
%
\begin{leftalign}
s &\in \mathbb{R} \\
v_n &\defas [s_1,s_2,\dots,s_n] \\
m_{n_1\times n_2} &\defas [[s_{11},s_{12},\dots,s_{1n_2}],\dots,[s_{n_11},s_{n_12},\dots,s_{n_1n_2}]]
\end{leftalign}
%
Expressions can be variables, literals, or operators:
%
\begin{leftalign}
x &\in \text{variables} \\
c &\defas s \alt v_n \alt m_{n_1\times n_2} \\
e &\defas c \alt
    \tau\;x=e \alt
    x=e \alt
    e_1+e_2 \alt
    e_1*e_2 \alt
    e_1\;\mathsf{.*}\;e_2 \alt
    \mathsf{dot}\ e
\end{leftalign}
%
The $.*$ operator represents elementwise multiplication (whereas $*$ is proper linear algebra multiplication).
Our type system consists of user-defined \emph{tags}, $T$, alongside built-in types:
%
\begin{leftalign}
n &\in \mathbb{N} \\
T &\in \text{tags} \\
\nu &\defas \top_n \alt
    \bot_n \alt T \\
\tau &\defas \mathsf{unit} \alt
    \mathsf{scalar} \alt
    \nu \alt
    \nu_1->\nu_2
\end{leftalign}
%
Function types $\nu_1 -> \nu_2$ are not ordinary functions; they are matrices used to map from one vector space to another using matrix--vector multiplication.  The top types $\top_n$ and $\top_{n\times m}$ can be written in code as $\mathsf{vec}n$ and $\mathsf{mat}n\mathsf{x}m$, respectively.  The bottom types $\bot_n$ and $\bot_{n\times m}$ are purely meant for use by the typechecker and so are not part of the surface syntax of the language.

\section{Subtype Ordering}
We define a judgment $\tau_1 \leq \tau_2$ for subtyping on ordinary types.
Subtyping is reflexive and transitive, as usual:
%
\begin{mathpar}
\inferrule
    { }
    {\tau\leq\tau}

\inferrule
    {\tau_1 \leq \tau_2 \and \tau_2 \leq \tau_3}
    {\tau_1 \leq \tau_3}
\end{mathpar}
%
The null type \textsf{unit} acts as a top type for all of $\tau$:
%
\begin{mathpar}
\inferrule
    { }
    {\tau\leq\mathsf{unit}}
\end{mathpar}

We also introduce a partial order on tags $\nu$, of the form $\Delta |- \nu_1 \leq_\Delta \nu_2$.  This ordering refers to a context $\Delta$, which is a map from tags $T$ to matrix types $\nu$:
%
\begin{mathpar}
\inferrule
    {\Delta|-\Delta(\nu_1)=\nu_2}
    {\Delta|-\nu_1\leq_\Delta \nu_2}
\end{mathpar}
%
This relation has the usual reflexivity and transitivity properties:
%
\begin{mathpar}
\inferrule
    { }
    {\Delta|-\nu\leq_\Delta \nu}

\inferrule
    {\Delta|-\nu_1 \leq_\Delta \nu_2 \and \Delta|-\nu_2 \leq_\Delta \nu_3}
    {\Delta|-\nu_1 \leq_\Delta \nu_3}
\end{mathpar}
%
Members of $\nu$, namely vectors and tagged vectors, have a top type representing general vectors of dimension $n$ as $\top_n$.  Additionally, we also introduce a bottom type $\bot_n$ for each vector dimension.  We naturally have the simple rule
%
\begin{mathpar}
\inferrule
    { }
    {\Delta|-\bot_n\leq_\Delta\top_n}
\end{mathpar}
%
These constructions require the following rules for any provided context $\Delta$
%
$$\Delta(\nu)\neq\bot_n
\qquad\textnormal{or}\qquad
\Delta(\top_n)\neq\nu$$
%
As a result of this construction, we can easily see that for any $\nu,\Delta$, there exists a $\top_n$ and $\bot_n$ such that
%
\begin{mathpar}
\inferrule
    { }
    {\Delta|-\bot_n\leq_\Delta\nu\qquad\Delta|-\nu\leq_\Delta\top_n}
\end{mathpar}
%
The function-like transformation matrix types, $\nu_1 -> \nu_2$, have the same standard subtyping relationship as ordinary function types:
%
\begin{mathpar}
\inferrule
	{\Delta|-\nu_1'\leq_\Delta \nu_1\qquad\Delta|-\nu_2\leq_\Delta \nu_2'}
	{\Delta|-\nu_1->\nu_2\leq_\Delta \nu_1'->\nu_2'}
\end{mathpar}
As with vectors, matrix functions have a $\top_{n_1\times n_2}$ and $\bot_{n_1\times n_2}$ for any pair of dimensions $n_1,n_2$.  These constructions are just maps $\nu_1->\nu_2$, namely:
%
\begin{mathpar}
\inferrule
	{}
	{\top_{n_1\times n_2}=\bot_{n_1}->\top_{n_2}}

\inferrule
	{}
	{\bot_{n_1\times n_2}=\top_{n_1}->\bot_{n_2}}
\end{mathpar}
%
These constructions similarly result in the fact that for any $\nu_1,\nu_2,\Delta$, there exists a $\top_{n_1\times n_2}$ and $\bot_{n_1\times n_2}$ such that:
%
$$\Delta|-\nu_1->\nu_2\leq_\Delta\bot_{n_1\times n_2}
\qquad\textnormal{or}\qquad
\Delta|-\top_{n_1\times n_2}\leq_\Delta\nu_1->\nu_2$$
%
\section{Static Semantics}

This typing judgement is a map from an expression to the type of that expression under some variable context $\Gamma$, from variable names to types, and some tag context $\Delta$ defined above.

\subsection{Subsumption}
Any type in a given context can be cast ``up'' at any time with one notable exception -- we impose a strictness condition on vectors which does not allow implicit subsumption into the top type $\top_n$.  This strictness allows the type system stronger control over regulating tag operation beyond simple dimension.  It is worth noting that strictness is unnecessary for implementing a sound type system, but does provide significant practical benefit in giving earlier compiler errors than would be otherwise observed.
%
\begin{mathpar}
\inferrule
	{\tau_1\leq\tau_2\qquad\Gamma,\Delta|-e:\tau_1,\Gamma}
	{\Gamma,\Delta|-e:\tau_2,\Gamma}

\inferrule
	{\Delta|-\nu_1\leq_\Delta \nu_2\qquad\Gamma,\Delta|-e:\nu_1,\Gamma}
	{\Gamma,\Delta|-e:\nu_2,\Gamma}
	\quad \nu_2\neq\top_n
\end{mathpar}

\subsection{Constants and Variable Declarations}
Declaring constants produce the types one would expect:
%
\begin{mathpar}
\inferrule
	{ }
	{\Gamma,\Delta|-():\mathsf{unit},\Gamma}

\inferrule
	{ }
	{\Gamma,\Delta|-s:\mathsf{scalar},\Gamma}
\end{mathpar}

Vector and matrix literals take on their respective bottom types of the appropriate dimension
%
\begin{mathpar}
\inferrule
	{ }
	{\Gamma,\Delta|-v_n:\bot_n,\Gamma}

\inferrule
	{ }
	{\Gamma,\Delta|-m_{n_1\times n_2}:\bot_{n_1\times n_2},\Gamma}
\end{mathpar}

Variables can be declared and assigned.  Assignment is required at declaration time.  Note that the entire variable must be reassigned in the case of vectors and matrices:
%
\begin{mathpar}
\inferrule
	{\Gamma,\Delta|-e:\tau,\Gamma'}
	{\Gamma,\Delta|-\tau\;x:=e:\mathsf{unit}, \Gamma', x \mapsto \tau}

\inferrule
	{\Gamma,\Delta|-\Gamma(x):\tau\qquad\Gamma,\Delta|-e:\tau,\Gamma'}
	{\Gamma,\Delta|-x:=e:\mathsf{unit}, \Gamma'}
\end{mathpar}

\subsection{Binary Operations}

All operators on scalars work as one might expect.

Types are closed under addition and scalar multiplication
%
\begin{mathpar}
\inferrule
	{\Gamma,\Delta|-e_1:\tau,\Gamma'\qquad\Gamma',\Delta|-e_2:\tau,\Gamma''}
	{\Gamma,\Delta|-e_1+e_2:\tau,\Gamma''}

\inferrule
	{\Gamma,\Delta|-e_1:\tau\qquad\Gamma,\Delta|-e_2:\mathsf{scalar}}
	{\Gamma,\Delta|-e_1*e_2:\tau}
\end{mathpar}

Component multiplication can be defined as a mathematical operation, but makes little formal sense.  Such operations are allowed, but don't interact with spaces and tags and result in a complete lack of information about a resulting matrix.
%
\begin{mathpar}
\inferrule
	{\Gamma,\Delta|-e_1:\top_n,\Gamma'\qquad\Gamma',\Delta|-e_2:\top_n,\Gamma''}
	{\Gamma,\Delta|-e_1\;\mathsf{.*}\;e_2:\top_n,\Gamma''}

\inferrule
	{\Gamma,\Delta|-e_1:\top_{n_1\times n_2},\Gamma'\qquad\Gamma',\Delta|-e_2:\top_{n_1\times n_2},\Gamma''}
	{\Gamma,\Delta|-e_1\;\mathsf{.*}\;e_2:\top_{n_1\times n_2},\Gamma''}
\end{mathpar}

Matrix multiplication is both a way of transforming from one tag to another and for composing two matrix functions together.  Note that these operations are NOT commutative and vectors are treated as column vectors with regards to matching dimensions.
%
\begin{mathpar}
\inferrule
	{\Gamma,\Delta|-e_1:\tau_1->\tau_2,\Gamma'\qquad\Gamma',\Delta|-e_2:\tau_1,\Gamma''}
	{\Gamma,\Delta|-e_1*e_2:\tau_2,\Gamma''}

\inferrule
	{\Gamma,\Delta|-e_1:\tau_1->\tau_2,\Gamma'\qquad\Gamma',\Delta|-e_2:\tau_2->\tau_3,\Gamma''}
	{\Gamma,\Delta|-\;e_1*e_2:\tau_1->\tau_3,\Gamma''}
\end{mathpar}

The dot product function requires that each component is of the same tag (note the implicit use of the strictness condition not allowing vectors to be implicitely cast to the top type to maintain this invariant).

\begin{mathpar}
\inferrule
	{\Gamma,\Delta|-e_1:\tau,\Gamma'\qquad\Gamma',\Delta|-e_2:\tau,\Gamma''}
	{\Gamma,\Delta|-\mathsf{dot}\;e_1\;e_2:\mathsf{scalar},\Gamma''}
\end{mathpar}

\section{Dynamic Semantics}

The operational semantics of this language map, in a single step, from a command to a constant and a state $\sigma$.  $\sigma$ itself is, as usual, a map from variable names to constants.

\subsection{Substitution}
Substitutions work as expected on the environment $\sigma$.
%
\begin{mathpar}
\inferrule
	{ }
	{\env{\tau\;x:=c}->(),\sigma[c/x]}

\inferrule
	{\env{e}->\env{e'}}
	{\env{\tau\;x:=e}->\env{\tau\;x:=e'}}
\end{mathpar}

\subsection{Mathematical Operations}
As usual, we can reduce on either side of the mathematical operations $\odot\in\{+,*,\mathsf{.*}\}$
%
\begin{mathpar}
\inferrule
	{\env{e_1}->\env{e_1'}}
	{\env{e_1\odot e_2}->\env{e_1'\odot e_2}}

\inferrule
	{\env{e_2}->\env{e_2'}}
	{\env{c\odot e_2}->\env{c\odot e_2'}}
\end{mathpar}

Addition and scalar multiplication have standard mathematical meaning.  Similarly, vector and matrix multiplication have standard mathematical meaning.  Component-wise multiplication is identical to matrix addition, except using multiplication of numbers in place of addition of numbers.  For readability, formal semantics of each of these operations will be ommitted.
\end{document}
